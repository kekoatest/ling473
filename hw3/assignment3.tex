\documentclass[11pt]{article}

\usepackage{fullpage}
\usepackage{forest}
\usepackage{tabto}
% \newcommand{\itab}[1]{\hspace{0em}\rlap{#1}}
% \newcommand{\tab}[1]{\hspace{.2\textwidth}\rlap{#1}}

% STYLE CHEAT SHEET
% \textit{}
% \textbf{}
% \tab{} \\start with one \tab{} per line
% ``quotations''
% \noindent

\begin{document}
\NumTabs{20}
\noindent
Kekoa Riggin\\
LING 473\\
HW 3

\section*{Question 1}
\noindent
*See end of document for tables of sample spaces\\

\noindent
For each of these questions, Bayes theorum should be applied. I was unable to produce the correct results with bayes and resorted to mapping out the sample space. 

\subsection*{Part a}
\noindent
P(X or Total = 7) = P(Dice roll) $ \times $ P(Second roll is difference of 7 and first roll)\\
P(X) = P(A) $ \times $ P(B)\\
P(X) = (6/6) $ \times $ (1/6)\\
P(X) = 6/36 = 1/6 = 1.6667\\


\subsection*{Part b}
\noindent
Bayes should be used here, but I used sample space\\
P(Y or total $>$= 9) = 10/36 = 0.27778

\subsection*{Part c}
\noindent
Bayes should be used here. I used sample space.\\
P(Z or Roll 1 greater than Roll 2) = 15/36 = 0.416667

\section*{Question 2}

\subsection*{Part a}
\noindent
158

\subsection*{Part b}
\noindent
P(. $|$nn) = 4/24

\subsection*{Part c}
\noindent
P(DT JJ) = 6/157

\subsection*{Part d}
\noindent
P(NN $|$ DT JJ) = 5/6

\subsection*{Part e}
\noindent
P(DT JJ $|$ NN) $ \times $ P(NN) = P(NN | DT JJ) $ \times $ P(DT JJ)\\
P(DT JJ $|$ NN) = $\frac{P(NN | DT JJ) \times  P(DT JJ)}{P(NN)}$\\
P(DT JJ $|$ NN) = $\frac{(5/6) \times (6/157)}{24/158}$\\
P(DT JJ $|$ NN) = 395/1884 = 0.20966

\section*{Question 3}
\noindent
The only open vowels in this set are \textit{gnat} and \textit{sand}. This means that there are 10 out of 12 vowels that are close. However, given the grouping of words to select from, the probability of choosing a close vowel rathern than an open one is not simply 10/12.\\\\
P(A) = 1/3\\
P(Close $|$ A) = 1/2\\\\
P(B) = 1/3\\
P(Close $|$ B) = 2/2\\\\
P(C) = 1/3\\
P(Close $|$ C) = 3/4\\\\
P(Close) = P(Close|A) + P(Close|B) + P(Close|C) = 9/12 = 0.75

\section*{Question 4}

\subsection*{Part a}
\noindent
The probability that we select a positive document from \textit{C} is 2/6. We must multiply this probability by the probability of this same document being seleceted from not \textit{C} in the next step in order to find the total probability of selecting a positive document.\\\\
P(X or Selecting a positive document from not C) = P(I Chooing positive doc we know is in not C) + (P(J Choosing doc we moved from C) $\times$ P(K Moving positive doc from C to not C)\\
P(X) = 1/3 + (1/3 $ \times $ 2/6) = 8/18 = 0.4444

\subsection*{Part b}
\noindent
For this question, we must use Bayes Theorum to find P(B|A) after already knowing P(A|B) from the previous question.\\
The probability of picking a positive doc, given that a positive do was moved from C to not C is 2/3 because that would make 2 positive docs and 1 not positive doc in set not C. The probability of moving a positive doc from C to not C is 1/3. This means our first equation is:\\\\
P(A$|$B)$ \times $P(B) = 2/3 $ \times$ 1/3 = 2/9\\\\
Knowing this, we can divide the left side of the equation by P(A), which we take from the question above, to find P(B$|$A):\\
$\frac{P(A|B)\times P(B)}{P(A)}$\\
$\frac{2/3\times 1/3}{8/18}$ = 1/2 = 0.5\\

\section*{Tables}

\noindent
\begin{table}[b]
\caption{Q1 Part a: Two Dice Total = 7}
  \begin{tabular}{|c|cccccc|}
  	\hline
    - & 1 & 2 & 3 & 4 & 5 & 6 \\
    \hline
    1 & - & - & - & - & - & X\\
    2 & - & - & - & - & X & -\\
    3 & - & - & - & X & - & -\\
    4 & - & - & X & - & - & -\\
    5 & - & X & - & - & - & -\\
    6 & X & - & - & - & - & -\\
    \hline
  \end{tabular}
\end{table}

\noindent
\begin{table}
\caption{Q1 Part b: Two Dice Total = 9}
  \begin{tabular}{|c|cccccc|}
  	\hline
    - & 1 & 2 & 3 & 4 & 5 & 6 \\
    \hline
    1 & - & - & - & - & - & -\\
    2 & - & - & - & - & - & -\\
    3 & - & - & - & - & - & X\\
    4 & - & - & - & - & X & X\\
    5 & - & - & - & X & X & X\\
    6 & - & - & X & X & X & X\\
    \hline
  \end{tabular}
\end{table}

\noindent
\begin{table}
\caption{Q1 Part c: Second Roll Higher than First}
  \begin{tabular}{|c|cccccc|}
  	\hline
    - & 1 & 2 & 3 & 4 & 5 & 6 \\
    \hline
    1 & - & X & X & X & X & X\\
    2 & - & - & X & X & X & X\\
    3 & - & - & - & X & X & X\\
    4 & - & - & - & - & X & X\\
    5 & - & - & - & - & - & X\\
    6 & - & - & - & - & - & -\\
    \hline
  \end{tabular}
\end{table}

\end{document}